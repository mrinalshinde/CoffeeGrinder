%DO NOT MESS AROUND WITH THE CODE ON THIS PAGE UNLESS YOU %REALLY KNOW WHAT YOU ARE DOING
\chapter*{Network Analysis}
\addcontentsline{toc}{chapter}{Network Analysis}

\noindent Network analysis is the method of planning and controlling projects by recording their interdependence in a diagrammatic form which enables each fundamental problem involved to be tackled separately. The objectives of network analysis are as follows:
\begin{itemize}
  \item To minimize idle resources
  \item To minimize the total project cost
  \item To trade off between time and cost of project
  \item To minimize production delays, interruptions and conflicts
  \item To minimize the total project duration
\end{itemize}

\noindent The network diagram is always drawn from left to right to show the stages of the project. The tasks highlighted in red display the critical path, while the blue tasks are not on the critical path. The critical path is the shortest time path through the network. The nodes represent the start, the duration and
the finishing of the individual task.

\begin{figure}[H]
\centering
{\includegraphics[ height=24cm, width=16cm]{full.png}}
\caption{Network Diagram}
\end{figure}

