%DO NOT MESS AROUND WITH THE CODE ON THIS PAGE UNLESS YOU %REALLY KNOW WHAT YOU ARE DOING
 \chapter*{Monitoring and Controlling}
\addcontentsline{toc}{chapter}{Monitoring and Controlling}

\noindent This process oversees all the tasks and metrics necessary to ensure that the approved project is within scope, on time, and on budget so that the project proceeds with minimal risk. This process involves comparing actual performance with planned performance and taking corrective action to yield the desired outcome when significant differences exist. 

\noindent Being the Project Manager, I will execute the project management plans and continuously manage and evaluate the overall project performance to provide confidence that the project will satisfy the relevant quality standards. I should continuously utlize the performance reporting as a necessary process for collecting and distributing performance information. This includes status reporting, progress measurement and forecasting. Using the WBS, Responsibility Matrix and Gannt Chart, I know how long each task should take and who handles which part of the project, so I can make sure those things are getting done. Clear expectations throughout the project will be communicated which will help the team stay on track and thus make monitoring easier. Weekly monitoring will be scheduled and the team is also expected to show the progress of that week in the weekly meetings. These meetings will be kept short and to the point to make them as efficient as possible. These meetings would also let us solve problems that arise.



